%!Tex program=xelatex
\documentclass[presentation,11pt]{beamer}
\usepackage{ctex}
%\usepackage{xeCJK}
\usepackage{graphics}



\usepackage{xcolor}
%\usepackage{fontspec}
%\usepackage{xunicode}
\usepackage{xltxtra}
%% 下面的包控制beamer的风格,可以根据自己的爱好修改
% \usepackage{beamerthemesplit}   % 使用split风格
% \usepackage{beamerthemeshadow}  % 使用shadow风格
% \usepackage[width=2cm,dark,tab]{beamerthemesidebar}



\usepackage{hyperref}       %为PDF文档创建超链接

\usefonttheme[onlymath]{serif}
\usepackage{amsmath}                   %数学符号与公式
\usepackage{amsfonts}                  %数学符号与字体


% 设置一些格式
%==================================================
\beamertemplateballitem
%\setbeamertemplate{theorems}[numbered]      %定理编号
%\setbeamertemplate{caption}[numbered]       %图和表格的标题显示编号
%\setbeamertemplate{navigation symbols}{}    %取消导航条
%==================================================




% 将默认的英文目录等改为中文,设置图号和公式号与章节对应,缩进大小
%==================================================
% \renewcommand{\contentsname}{\LARGE{\bf{目 \quad 录}}}
% \renewcommand{\abstractname}{\Large{\bf{摘 \quad 要}}}
% \renewcommand{\appendixname}{\Large{\bf{附 \quad 录}}}
% \renewcommand{\theequation}{\arabic{section}.\arabic{equation}}      %公式号与章节对应
% \renewcommand{\figurename}{\normalsize{图 \arabic{section}.\arabic{figure}}}   %改figure为图
% \makeatletter
% \renewcommand{\fnum@figure}[1]{\textbf{\figurename~}\hspace{10pt} \sffamily}   %图号与章节对应
% \makeatother
\setlength{\parindent}{2em}       %设置缩进为两个大写M的宽度,大约为两个汉字的宽度
%==================================================


% 设置英文字体
%==================================================


%\usefonttheme{professionalfonts}


%\defaultfontfeatures{Scale=MatchLowercase} % 这个参数保证 serif、sans-serif 和 monospace 字体在小写时大小匹配
\setmainfont[Mapping=tex-text]{Times New Roman} % 使用 XeTeX 的 text-mapping 方案,正确显示 LaTeX 样式的双引号(`` '')
% \setmainfont[Mapping=tex-text]{Palatino Linotype}
%\setsansfont[Mapping=tex-text]{Arial}
%\setsansfont{微软雅黑}
 %\setsansfont[Mapping=tex-text]{DejaVu Sans YuanTi}
%\setmonofont{Courier New}
 %\setmonofont{Monaco}
% \setmonofont{DejaVu Sans YuanTi}
%==================================================


%\mode<presentation>
%{
%  \usetheme{Warsaw}
%  % 可以改成别的主题
%  \setbeamercovered{transparent}
%  % 也可删去
%}
\setbeamertemplate{caption}[numbered]   %beamer默认的图片的caption不带编号


%使用metapost动画
%\usepackage{xmpmulti}
%\DeclareGraphicsRule{*}{mps}{*}{}

%\setmainfont{AdobeKaitiStd}
%\setmainfont{AdobeKaitiStd}
%\setsansfont{SimHei}

%\setsansfontWenQuanYi Zen Hei}     %Linux下一般没有SimHei,使用文泉驿正黑或其他系统中可用的字体代替

\usecolortheme[named=green]{structure}	
%\usetheme{Warsaw}
%\usetheme{AnnArbor}
%\usetheme{PaloAlto}
%\usetheme{Antibes}
%\usetheme{CambridgeUS}
%\usetheme{Copenhagen}
%\usetheme{Darmstadt}
%\usetheme{Dresden}
%\usetheme{Frankfurt}
%\usetheme{Goettingen}
%\usetheme{Hannover}
%\usetheme{Ilmenau}
%\usetheme{JuanLesPins}
%\usetheme{Luebeck}
\usetheme{Madrid}
%\usetheme{Marburg}
%\usetheme{PaloAlto}
%\usetheme{Singapore}
%\usetheme{Warsaw}


\title{beamer测试}
\subtitle{Beamer中英文混排测试}
\author{pipjing}
\institute{\kaishu 西南大学\  数院}
\date{\today}


% 如果你想要在每一小节之前都显示一下目录, 则可把一下小段的注解号 "%" 删去
%\AtBeginS section[]
%{
%  
%\begin{frame}<beamer> 
%    \frametitle{概要} 
%    \tableofcontents[currentsection,currents section] 
%  \end{frame}

%}


% 除掉以下命令的注解 "%" 后, 许多环境都会自动逐段显示
%\beamerdefaultoverlayspecification{<+->}


%索引使用中文
%\renewcommand\contentsname{目\ 录}
%\renewcommand\listfigurename{插图目录}
%\renewcommand\listtablename{表格目录}
%%\renewcommand\abstractname{摘\ 要} %err undefined
%%\renewcommand\refname{参考文献}         %article类型
%\renewcommand\bibname{参\ 考\ 文\ 献}    %book类型
%\renewcommand\indexname{索\ 引}
%\renewcommand\figurename{图}
%\renewcommand\tablename{表}
%\renewcommand\partname{部分}




%定制定理环境
%\newtheorem{mythl}{引理}[section]
%\newtheorem{mytht}{定理}[section]
%\newtheorem{mythr}{注}[section]
%\newtheorem{mythc}{推论}[section]
%\newtheorem{mythd}{定义}[section]
%\newtheorem{mytha}{公理}
%\newtheorem{mythp}{命题}
%\newtheorem{mythe}{练习}
%\newtheorem{myli}{例}[section]




\begin{document} 
 
 
\XeTeXlinebreaklocale "zh"  % 表示用中文的断行 
\XeTeXlinebreakskip = 0pt plus 1pt % 多一点调整的空间 
 
 
\begin{frame} 
\titlepage 
 
 
\end{frame} 
 
 
\begin{frame} 
\frametitle{目录} 
    \tableofcontents     % 你也可以插入选项 [pausesections] 
 
 
\end{frame} 
 
 
 
 
\section{常用功能测试} 
\subsection{简单格式测试} 
 
 
\begin{frame} 
   \frametitle{常用中文字体测试} 
 
 
 
 
\kaishu 中文显示正确!\\  
\songti 中文显示正确!\\  
\heiti 中文显示正确!\\  
\fangsong 中文显示正确!\\ 


% English test succeed!\\ 
% \begin{align}\label{eq:ei} 
%     E=mc^2\\ 
%     E=\hbar \nu 
% \end{align} 
 
 
\end{frame} 
 
\begin{frame}[allowframebreaks] 
\frametitle{大段中文测试} 
教育学是一门独立的学科。教育学是研究人类教育现象和解决教育问题、揭示一般教育规律的一门社会科学。教育是广泛存在于人类生活中的社会现象,教育学是有目的地培养社会人的活动。它是通过对各种教育现象和问题的研究揭示教育的一般规律。
19世纪中叶以后,马克思主义的产生,近代心理学、生理学的发展,为科学化教育奠定了辩证唯物主义哲学和自然科学基础。现代生产和科学技术的发展,教育实践的广泛性、丰富性,更进一步推动了教育学的发展。教育学的研究对象是人类教育现象和问题,以及教育的一般规律。是教育、社会、人之间和教育内部各因素之间内在的本质的联系和关系,具有客观性、必然性、稳定性、重复性。如教育与社会的政治、生产、经济、文化、人口之间的关系,教育活动与人的发展之间的关系,教育内部的学校教育、社会教育、家庭教育之间的关系,小学教育、中学教育、大学教育之间的关系,中学教育中教育目标与教学、课外教育之间的关系,教育、教学活动中智育与德、体、美、劳诸育之间的关系,智育中教育者的施教与受教育者的受教之间的关系,学生学习活动中学习动机、学习态度、学习方法与学习成绩之间的关系等等都存在着规律性联系。教育学的任务就是要探讨、揭示种种教育的规律,阐明各种教育问题,建立教育学理论体系。
\end{frame} 

\begin{frame}
  \frametitle{大段英文测试}
  What is GitHub Pages?
GitHub Pages is a static site hosting service.

GitHub Pages is designed to host your personal, organization, or project pages directly from a GitHub repository. To learn more about the different types of GitHub Pages sites, see "User, organization, and project pages."

You can create and publish GitHub Pages online using the Automatic Page Generator. If you prefer to work locally, you can use GitHub Desktop or the command line.

\end{frame}


\begin{frame}
  \frametitle{数学公式测试}
   \begin{eqnarray*}
                \vec{x}\stackrel{\mathrm{def}}{=}{x_1,\dots,x_n}\\
                {n+1 \choose k}={n \choose k}+{n \choose k-1}\\
                \sum_{k_0,k_1,\ldots>0 \atop k_0+k_1+\cdots=n}A_{k_0}A_{k_1}\cdots
           \end{eqnarray*}

\end{frame}
 
 
\subsection{图片测试} 
 
 
 
\section{其他测试} 
 
 
\begin{frame} 
\frametitle{my first beamer!} 
\begin{definition} 
a \alert{beamer} is to create a slides show! 
\end{definition} 
\end{frame} 
 

 \subsection{换页动态效果}

 \begin{frame}[<+->]
   \frametitle{\subsecname}
  在全屏下才会出现效果。
  \begin{enumerate}
  \item 水平出现效果
    \transblindshorizontal<1>
  \item 竖直出现效果
    \transblindsvertical<2>
  \item 从中心到四角
    \transboxin<3>
  \item 从四角到中心
    \transboxout<4>
  \item 溶解效果
    \transdissolve<5>
  \item Glitter
    \transglitter<6>
  \item 竖直撕开(向内)
    \transsplitverticalin<7>
  \item 竖直撕开(向外)
    \transsplitverticalout<8>
  \item 涂抹
    \transwipe<9>
  \item 渐出
    \transduration<10>{1}
  \end{enumerate}

 \end{frame}
  
 
\section{总结一下} 
 
 
\begin{frame} 
    \frametitle{为什么使用Beamer?} 
    \begin{enumerate} 
      \item \XeLaTeX 非常棒! 
      \begin{enumerate} 
        \item 可以使用系统字体; 
        \item 可以中英文混排; 
        \item 这些优点已经足够了! 
      \end{enumerate} 
      \item Beamer非常棒! 
      \item \LaTeX 非常棒! 
    \end{enumerate} 
\end{frame} 
 


 
 
\end{document}

